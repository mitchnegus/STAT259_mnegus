l\documentclass{report}
% PACKAGES %
\usepackage[english]{} % Sets the language
\usepackage[margin=2cm]{geometry} % Sets the margin size
\usepackage{fancyhdr} % Allows creation of headers
\usepackage{xcolor} % Allows the use of color in text
\usepackage{float} % Allows figures and tables to be floats
\usepackage{appendix}
\usepackage{amsmath} % Enhanced math package prepared by the American Mathematical Society
	\DeclareMathOperator{\sech}{sech} % Include sech
\usepackage{amssymb} % AMS symbols package
\usepackage{mathrsfs}% More math symbols
\usepackage{bm} % Allows you to use \bm{} to make any symbol bold
\usepackage{bbold} % Allows more bold characters
\usepackage{verbatim} % Allows you to include code snippets
\usepackage{setspace} % Allows you to change the spacing between lines at different points in the document
\usepackage{parskip} % Allows you alter the spacing between paragraphs
\usepackage{multicol} % Allows text division into multiple columns
\usepackage{units} % Allows fractions to be expressed diagonally instead of vertically
\usepackage{booktabs,multirow,multirow} % Gives extra table functionality
\usepackage{hyperref} % Allows hyperlinks in the document
\usepackage{rotating} % Allows tables to be rotated
\usepackage{graphicx} % Enhanced package for including graphics/figures
	% Set path to figure image files
	%\graphicspath{ } }
\usepackage{listings} % for including text files
	\lstset{basicstyle=\ttfamily\scriptsize,
        		  keywordstyle=\color{blue}\ttfamily,
        	  	  stringstyle=\color{red}\ttfamily,
          	  commentstyle=\color{gray}\ttfamily,
          	 }		
\newcommand{\tab}{\-\hspace{1cm}}

% Create a header w/ Name & Date
\pagestyle{fancy}
\rhead{\textbf{Mitch Negus} \; 9/1/2017}

\begin{document}
\thispagestyle{empty}
\sffamily

\large {STAT259 Summary {1} \hfill Mitch Negus\\
		\hspace*{\fill} 9/7/2017\\ }
\section*{\textsf{Developing open source scientific practice \\ \normalsize Millman, K. and Perez F.}}

--COMPUTATIONAL SCIENCE--\\
\-\\
REPRODUCIBLE SCIENCE\\
- for good science, it must be reproducible\\
- open source software community has developed this framework/workflow\\
- though feasibility of reproduction may be limited, the possibility of reproduction must be demanded\\
\-\\
\-\\
LIFE CYCLE\\
*individual\\
*collaboration\\
*production\\
*publication\\
*education\\
\-\\
-Most common tools create discontinuities across stages of workflow\\
-Results are considered separate from process, rather than a unified science product\\
-Joining tools and stages requires both technical and social changes\\
\-\\
Reproducibility must be a commitment from the start\\
\-\\
\-\\
OPEN SOURCE\\
-moving science forward requires computational literacy--> science is open source --> contributing to science will require open source practices\\
challenges exist with making science fully open source (author recognition, first to publish, etc.)\\
\-\\
\-\\
\-\\
--PRACTICE--\\
\-\\
VERSION CONTROL\\
-files stored in repositories, require commits (w/ message)\\
-allow branching and merging\\
-modern systems allow data integrity verification (cryptographically fingerprinting)\\
-also limited when dealing w/ large binary files (solutions being developed in this regard)\\ 
\-\\
AUTOMATED EXECUTION\\
-reproducibility should extend to process; best to automate all steps when possible
-still should be able to be understood by people\\
-make files facilitate this process\\
\-\\
TESTING
-testing should accompany product development (test-driven-design)
-allows focus on use (rather than details)
-TDD prevents "getting lost" in tangled code web
\-\\
READABILITY\\
-you and others will read your code (esp. to verify results)\\
-self-doc code reduces external documentation by being clear and forward\\
-use the right level of abstraction when writing mathematical expressions (don't simply too much, but don't avoid it entirely)\\
-comments may be uncoupled from code (one changes, the other is not updated)\\
-use of docstrings allows coupling of docs to code--> then autogeneration for web\\
\-\\
INFRASTRUCTURE\\
-hosted version control allows group collaboration\\
-continuous integration to automatically execute test-suite\\
\-\\
PULL REQUEST\\
-pull request akin to peer review\\
-anyone can chime in, lasting document of decisions\\
-private branches (maintain credit, history, and privacy while allowing transparency after integration\\

-linear algebra book by Rob Beezer (U. Puget Sound)
\end{document}