\documentclass{report}
% PACKAGES %
\usepackage[english]{} % Sets the language
\usepackage[margin=2cm]{geometry} % Sets the margin size
\usepackage{fancyhdr} % Allows creation of headers
\usepackage{xcolor} % Allows the use of color in text
\usepackage{float} % Allows figures and tables to be floats
\usepackage{appendix}
\usepackage{amsmath} % Enhanced math package prepared by the American Mathematical Society
	\DeclareMathOperator{\sech}{sech} % Include sech
\usepackage{amssymb} % AMS symbols package
\usepackage{mathrsfs}% More math symbols
\usepackage{bm} % Allows you to use \bm{} to make any symbol bold
\usepackage{bbold} % Allows more bold characters
\usepackage{verbatim} % Allows you to include code snippets
\usepackage{setspace} % Allows you to change the spacing between lines at different points in the document
\usepackage{parskip} % Allows you alter the spacing between paragraphs
\usepackage{multicol} % Allows text division into multiple columns
\usepackage{units} % Allows fractions to be expressed diagonally instead of vertically
\usepackage{booktabs,multirow,multirow} % Gives extra table functionality
\usepackage{hyperref} % Allows hyperlinks in the document
\usepackage{rotating} % Allows tables to be rotated
\usepackage{graphicx} % Enhanced package for including graphics/figures
	% Set path to figure image files
	%\graphicspath{ } }
\usepackage{listings} % for including text files
	\lstset{basicstyle=\ttfamily\scriptsize,
        		  keywordstyle=\color{blue}\ttfamily,
        	  	  stringstyle=\color{red}\ttfamily,
          	  commentstyle=\color{gray}\ttfamily,
          	 }		
\newcommand{\tab}{\-\hspace{1cm}}

% Create a header w/ Name & Date
\pagestyle{fancy}
\rhead{\textbf{Mitch Negus} \; 9/25/2017}

\begin{document}
\thispagestyle{empty}
\sffamily

\large {STAT259 Summary {3} \hfill Mitch Negus\\
		\hspace*{\fill} 9/25/2017\\ }
\section*{\textsf{Treating Code as an Essay \\ \normalsize Matsumoto, Y.}}
\section*{\textsf{Python’s Dictionary Implementation: Being All Things to All People \\ \normalsize Kuchling, A.}}

\textbf{Summary}\\
\tab ...
\-\\
\textbf{Exploration}\\
\tab ...

\newpage

\textbf{Notes}\\

-----Code as Essay-----\\
- code will most likely need to be rewritten (debugged, updated, etc.); people must be able to read the code or else these functions will not be realized (even the best computer can't read code that hasn't been updated properly)
- computers can deal with complexity; people can't \\
- concise is important in writing (don't waste words); similarly, brevity is important in code (be to the point)\\
	\tab -allows scanning of code by eye to find utility \\
	\tab -redundancy reduces cost of maintaining consistency (do it once, fix it once) \\
- code should be kept as simple as possible (I agree with this statement, though I think there comes a limit where simplicity and functionality converge)\\
	\tab $\rightarrow$ classes are significantly more complex than functions or simple scripts, but are drastically more powerful; know when to use both\\
	\tab $\rightarrow$ avoiding redundancy can lead to complicated software structures; simplicity should be balanced against abstraction

\-\\
\-\\
\-\\
-----Python Dictionaries-----\\
- ...





\end{document}